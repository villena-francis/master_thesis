\chapter{Discussion}

\section{Computational and Resource Requirements}

The synthetic data analysis configuration in this project demanded substantial 
computational resources: 14 TB of storage (2.73\% of CNIO's HPC cluster 
capacity) and 832 cores for VISOR-LASeR execution (52 cores per sample, 
exceeding 114\% of available cores). Furthermore, the single-job-per-GPU 
restriction for SVision-pro execution led to extended processing times. This 
intensive resource utilization presented significant challenges given the 
cluster's shared nature among multiple CNIO research groups and projects.
These findings emphasize that computational tool selection should consider not 
only accuracy but also efficient resource management based on available 
computing resources as critical evaluation criteria.

\section{Simulation Design and Parameters}

The selection of SVs for simulation was influenced by ongoing research in the 
CNIO Bioinformatics Unit's long-reads group, specifically their collaboration 
with Hospital 12 de Octubre's Hematological Malignancies group on WGS of MM 
patient samples, comparing pre-treatments and relapses. Consequently, the SV 
set includes both characteristic MM variants and more speculative cancer-related 
SVs to cover a broader spectrum of SVs. Despite the availability of 
T2T genomes, GRCh38 was chosen as the reference genome due to its extensive use 
in research and the substantial accumulation of annotation-associated findings 
throughout its trajectory.

The SV simulation was configured to represent bulk sequencing of a homogeneous 
cell population containing a single clone, with all variant allele frequencies 
(VAFs) set to 0.5. This configuration ensures variants are present on one allele 
and represented in half of the generated sequencing reads. Such design aimed to 
provide clear variant representation in the reads, theoretically ensuring 
reliable detection by callers while facilitating visual validation through GW. 
However, this idealized scenario differs from real tumor samples, where variant 
allele frequencies can vary significantly due to tumor heterogeneity and normal 
cell contamination \cite{dagogo-jack_tumour_2018}.

Reads were simulated with a mean length of 15,000 bp and a standard deviation of 
13,000 bp, using VISOR-LASeR default settings. These parameters align with 
realistic values obtainable through actual sequencing using the PromethION 
platform's protocol for 10 kb human DNA with Ligation Sequencing Kit V14 
\cite{noauthor_ligation_2022}, which has been selected for sequencing the MM
patient samples. Notably, this protocol aims to generate $\sim$30-40x genome 
coverage. Our benchmarking results at higher coverages (100x and 200x) showed 
no substantial performance improvements for any caller in detecting simulated 
SVs, suggesting that the additional costs and efforts associated with using 
multiple flow cells and preparing larger sample quantities may not be justified. 
However, Severus detected unplanned VNTR anomalies at lower coverages (30x and 50x), 
introduced by the Badread error model trained on ONT R10.4.1 reads. This 
observation is crucial for real sample sequencing: while such artifacts were 
more prevalent in R9.4 reads (see \textbf{Figure \ref{fig:ont_pores}}), 
they might still influence SV calling results, as even a small number of reads 
containing these sequencing artifacts could be considered representative by 
callers at the coverage levels currently achievable with a single flowcell.

\section{SV Caller Performance and Limitations}

Performance metrics position SAVANA as the top performer, yet its limitation to 
breakpoint correlation without SV classification represents a significant 
drawback. This limitation, combined with substantially higher RAM consumption 
and execution times compared to other callers, impacts its practical utility. 
Severus, despite slightly lower precision, matches SAVANA's recall while 
maintaining lower RAM usage and provides comprehensive SV classification 
alongside visual representations of chromosomal rearrangements. These 
characteristics likely influenced Severus's integration into the latest EPI2ME 
release, Oxford Nanopore's open-source platform designed to provide wet lab 
scientists with a user-friendly interface for data analysis without requiring 
advanced bioinformatics skills. However, despite its integration, EPI2ME's 
documentation lacks comparative analyses justifying Severus's selection over 
other SV callers, possibly due to its target audience 
\cite{oxford_nanopore_technologies_epi2me_nodate}.

Our evaluation employed minimum argument sets for all callers, relying on 
developer-configured default parameters for complex adjustments. This 
standardized approach may explain the poor performance of Sniffles2 and 
SVision-pro, particularly in detecting large SVs, as default settings might not 
be optimized for such variants. Notably, following our analysis, Sniffles2 
version 2.5 released improvements specifically targeting detection of large 
deletions and duplications ($>$ 50 kb) \cite{noauthor_releases_nodate}. 
Interestingly, the better-performing tools in our analysis (SAVANA and Severus) 
remain available only in preprint servers, while Sniffles2 and SVision-pro are 
published in peer-reviewed journals.

\section{Clinical Relevance and Technical Challenges}

VISOR toolkit proved valuable for evaluating SV callers' capability to identify 
characteristic large structural events in Multiple Myeloma using ONT long reads, 
particularly those with diagnostic significance. Through synthetic data 
generation, we successfully validated the detection of two of the most frequent 
chromosomal aberrations in Multiple Myeloma, notably the largest SVs in our 
simulation set, using SAVANA and Severus: the tandem amplification of 1q21+ 
(100 Mb) and an IGH-involving translocation (1,5 Mb). These structural variants, 
currently verified in clinical settings through FISH due to the limitations of 
short-read sequencing assembly, represent critical diagnostic markers that could 
potentially be identified through long-read sequencing approaches.

The inability to simulate certain SVs provides valuable insights into technical 
limitations. While VISOR-HACk module unambiguously inserts all chromosomal 
abnormalities into a FASTA file using BED-formatted instructions and the 
reference genome as a template, challenges emerge in the VISOR-LASeR module,
which generates and aligns reads to the reference genome using minimap2. These 
limitations manifest in two ways: standard read lengths may be insufficient for 
reconstructing certain events, suggesting the potential need for ultra-long read 
protocols capable of generating sequences up to 4 Mb, and minimap2 alignment 
accuracy may be compromised for complex variants. For instance, in the 
copy-paste translocation designed to generate a proximal KRAS duplicate, the 
aligner appears to have defaulted to mapping all reads to the original gene 
position. Similarly, for the cut-paste translocation, only the deletion 
component was detected, while the inverted sequence insertion failed to be 
properly positioned, leaving its corresponding reads unaccounted for in the 
alignment.

\section{Visualization Tools and Challenges}

Initial visualization attempts of synthetic data were made using the 
widely-adopted Integrative Genomics Viewer (IGV) version 2.17.3. However, 
despite using chromosome-specific BAM files generated through the ``bam-splitter'' 
workflow, IGV's performance proved inadequate, exhibiting slow loading times and 
frequent crashes. Through email correspondence, Severus's lead developer shared 
similar experiences with long-read BAM files in IGV, suggesting a workaround of 
generating smaller BAM files containing only the SV regions with 1 Mb upstream 
and downstream sequences. While this approach would have been feasible for 
planned SVs, it proved impractical for investigating additional findings like 
VNTR anomalies due to automation limitations.

GW emerged as a capable alternative, efficiently handling both individual 
chromosome and whole-genome BAM files. Its VCF compatibility created an 
interactive index for rapid SV coordinate navigation, dramatically accelerating 
event verification. However, GW requires terminal-based operation in conjunction 
with an alignment view display, demanding more advanced computational skills 
compared to IGV's graphical user interface \textbf{Figure~\ref{fig:GW_UI}}. 
Nevertheless, GW's developer demonstrated strong responsiveness to error reports 
and feature requests through Github. Our project-specific experience led to 
reporting Conda installation \cite{noauthor_error_nodate} and loading genome
anotation files \cite{noauthor_display_nodate} issues, and requesting vector 
format export capabilities for alignment visualizations 
\cite{noauthor_saving_nodate}.

\section{Future Directions}

The development of robust tools for SV analysis in cancer genomes requires 
extensive testing and validation. While this study evaluated SV detection 
capabilities using synthetic data across different sequencing coverages, 
it primarily demonstrates the value of generating simulated datasets for 
benchmarking existing tools, developing new algorithms, and training machine 
learning models. These computational advances ultimately contribute to 
improving cancer genomics analysis in clinical settings.

A primary challenge lies in identifying structural events that, despite 
existing in the template genome, remained undetected in alignments. The 
unaligned reads in FASTQ format will be valuable for expanding benchmarks to 
include long-read aligners beyond minimap2, enabling more comprehensive tool 
evaluation. This expansion is crucial for improving the reliability of genomic 
analysis in clinical settings.

To enhance the exploration of SVs in cancer, future work should investigate 
several key aspects. First, examining the impact of read length on Sniffles2 and 
SVision-pro's calling capabilities, starting with sizes successfully identified 
in their published work. Second, introducing tumor heterogeneity through 
multiple clone combinations in synthetic sequencing data would better reflect 
real tumor complexities and allow evaluation of how varying VAFs affect SV 
detection capabilities. Third, exploring ultra-long read protocol parameters 
could help reconstruct previously undetected SVs. Based on our findings and 
resource efficiency considerations, these tests could focus on 30x and 50x 
sequencing coverages, aligning with practical clinical sequencing depths.

Data collection automation represents another critical area for improvement in 
clinical implementation. Currently, computational performance statistics are 
manually extracted from Slurm logs, a time-consuming process that could be 
streamlined through automated scripts. In anticipation of future evaluations, 
we have requested VISOR toolkit enhancements to log all parameters, both 
specified and default, facilitating automated tracking of simulation 
characteristics \cite{noauthor_add_nodate}. Additionally, The implementation of 
Truvari, a comprehensive toolkit for benchmarking, merging, 
and annotating structural variants from VCF files, would enhance the workflow
\cite{english_truvari_2022}. Its capabilities make it 
particularly suitable for analyzing tumor genome evolution through longitudinal 
sequencing and SV calling analysis.

Continuous communication with tool developers remains essential for optimizing 
clinical applications, sharing synthetic data experiences, and facilitating 
potential improvements. In this context, frequent requests for GW enhancements, 
particularly regarding data visualization and high-quality figure export 
capabilities, will support better clinical result interpretation and 
documentation.

Most critically, the experience gained through synthetic data analysis must be 
validated on ONT-sequenced tumor-normal paired patient samples, integrating 
structural variation analysis with SNVs and methylation data. This comprehensive 
genomic characterization approach aims to provide clinicians with more accurate 
and complete information for personalized cancer treatment decisions, ultimately 
improving patient outcomes through better-informed therapeutic strategies.
