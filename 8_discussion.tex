\chapter{Discussion}

\section{Computational and Resource Requirements}

The synthetic data analysis configuration in this project demanded substantial 
computational resources: 14 TB of storage (2.73\% of CNIO's HPC cluster 
capacity) and 832 cores for VISOR-LASeR execution (52 cores per sample, 
exceeding 114\% of available cores). Furthermore, the single-job-per-GPU 
restriction for SVision-pro execution led to extended processing times. This 
intensive resource utilization presented significant challenges given the 
cluster's shared nature among multiple CNIO research groups and projects.
These findings emphasize that computational tool selection should consider not 
only accuracy but also efficient resource management based on available 
computing resources as critical evaluation criteria.

% TB of storage (8\% of CNIO's HPC cluster capacity) and 832 cores for VISOR-LASeR 
% execution (52 cores per sample, exceeding 114\% of available cores). 
% Furthermore, the single-job-per-GPU restriction for SVision-pro execution led to 
% extended processing times. This intensive utilization presented significant 
% challenges given the cluster's shared nature among multiple CNIO research groups 
% and projects. These findings emphasize that computational tool selection should 
% consider not only accuracy but also efficient resource management as critical 
% evaluation criteria.

% Los datos sintéticos generados han requerido 40 TB de almacenamiento, lo cual 
% supone alrededor del 8\% del volumen de almacenamiento disponible del HPC 
% cluster del CNIO. En relación a los recursos de cómputo, la ejecución de 
% VISOR-Laser para generar el dataset de 16 ficheros BAM demanda el uso de 832 
% cores (52 por muestra), lo cual supone el 114\% de los disponibles en el cluster.
% Todo ello, sumado a que el sistema de gestión de trabajos impide la ejecución 
% de más de un trabajo en cada GPU (usada en este caso para la ejecución de 
% SVision-pro) suponen una elevado demanda de unos recursos computacionales
% disponibles y demandados por grupos y proyectos de investigación de todo el CNIO. 

\section{Simulation Design and Parameters}

The selection of SVs for simulation was influenced by ongoing research in the 
CNIO Bioinformatics Unit's long-reads group, specifically their collaboration 
with Hospital 12 de Octubre's Hematological Malignancies group on WGS of MM 
patient samples, comparing pre-treatments and relapses. Consequently, the SV 
set includes both characteristic MM variants and more speculative cancer-related 
SVs to cover a broader spectrum of SVs. Despite the availability of 
T2T genomes, GRCh38 was chosen as the reference genome due to its extensive use 
in research and the substantial accumulation of annotation-associated findings 
throughout its trajectory.

The SV simulation was configured to represent bulk sequencing of a homogeneous 
cell population containing a single clone, with all variant allele frequencies 
(VAFs) set to 0.5. This configuration ensures variants are present on one allele 
and represented in half of the generated sequencing reads. Such design aimed to 
provide clear variant representation in the reads, theoretically ensuring 
reliable detection by callers while facilitating visual validation through GW. 
However, this idealized scenario differs from real tumor samples, where variant 
allele frequencies can vary significantly due to tumor heterogeneity and normal 
cell contamination \cite{dagogo-jack_tumour_2018}.

Reads were simulated with a mean length of 15,000 bp and a standard deviation of 
13,000 bp, using VISOR-LASeR default settings. These parameters align with 
realistic values obtainable through actual sequencing using the PromethION 
platform's protocol for 10 kb human DNA with Ligation Sequencing Kit V14 
\cite{noauthor_ligation_2022}, which has been selected for sequencing the MM
patient samples. Notably, this protocol aims to generate $\sim$30-40x genome 
coverage. Our benchmarking results at higher coverages (100x and 200x) showed 
no substantial performance improvements for any caller in detecting simulated 
SVs, suggesting that the additional costs and efforts associated with using 
multiple flow cells and preparing larger sample quantities may not be justified. 
However, Severus detected unplanned VNTR anomalies at lower coverages (30x and 50x), 
introduced by the Badread error model trained on ONT R10.4.1 reads. This 
observation is crucial for real sample sequencing: while such artifacts were 
more prevalent in R9.4 reads (see \textbf{Figure \ref{fig:ont_pores}}), 
they might still influence SV calling results, as even a small number of reads 
containing these sequencing artifacts could be considered representative by 
callers at the coverage levels currently achievable with a single flowcell.

% Sin embargo, 
% Severus detectó anomalías no planificadas dentro de VNRTs en las coverages más 
% bajas (30x y 50x), añadidas por el modelo de error de Badread entrenado con 
% trained on ONT R10.4.1 reads.

% las lecturas fueron simuladas a una longitud promedio de 15000 pb y una 
% desviación estandar de 13000, valores por defecto de VISOR-LASeR. Estos valores 
% entran dentro de lo razonablemente obtenible con secuenciación real with the 
% protocol for sequencing 10 kb human DNA on PromethION using the Ligation 
% Sequencing Kit V14. Dicho kit es el seleccionado para la secuenciación de las
% muestras de pacientes con Mieloma Múltiple.

% Para la simulación de las SVs se ajustó que las muestras correspondiesen a la 
% secuenciación bulk de una población celular compuesta por un solo clon cuyas 
% variant allele frequency son todas iguales a 0.5, lo cual se traduce en que 
% las variante está presente en uno de los dos alelos así como en la mitad de 
% de las lecturas de secuenciación generadas. Con esta configuración se buscaba 
% que las SVs introducidas tuviesen una representatividad en las lecturas bastante 
% evidente para los callers, la cual teoricamnete garantizaría su identificación
% a la par de que facilitar enormemente comprobarlas visualmente a través de GW.

% Es necesario tener en cuenta esto a la hora de 
% secuenciar muestras reales, este tipo de artefactos en muestras reales 
% eran mucho más frecuentes en R9.4 reads pero podrían seguir apareciendo en los
% SV callings debido a que en el rango de coverages que se pueden actualmente
% obtener con una flowcell pocas lecturas con estos artefactos de secuenciación 
% pueden ser representativas para el caller.

% La selección de las SVs para las simulaciones viene influenciada por la línea
% de trabajo del grupo de trabajo de long-reads of CNIO Bioinformatics Unit
% durante el desarrollo de este proyecto, concretamente a Set of pre-treatments 
% and relapses WGS samples from Multiple Myeloma (MM) patients en colaboración con 
% el grupo de Hemato Malignancies del Hospital 12 de Octubre. Es por ello que el 
% set de SVs incluye algunas típicas de MM así como otras relacionadas con 
% cáncer más especulativas con las que cubrir otros tipos de SVs. También es la
% razón por la que a pesar de la existencia de genomas T2T, se trabajó con el 
% genoma de referencia GRCh38 que sigue siendo la opción preferente en 
% investigación debido a que en su larga trayectoria se han acumulado gran 
% cantidad de resultados asociados a las anotaciones de esta referencia.

\section{SV Caller Performance and Limitations}

Performance metrics position SAVANA as the top performer, yet its limitation to 
breakpoint correlation without SV classification represents a significant 
drawback. This limitation, combined with substantially higher RAM consumption 
and execution times compared to other callers, impacts its practical utility. 
Severus, despite slightly lower precision, matches SAVANA's recall while 
maintaining lower RAM usage and provides comprehensive SV classification 
alongside visual representations of chromosomal rearrangements. These 
characteristics likely influenced Severus's integration into the latest EPI2ME 
release, Oxford Nanopore's open-source platform designed to provide wet lab 
scientists with a user-friendly interface for data analysis without requiring 
advanced bioinformatics skills. However, despite its integration, EPI2ME's 
documentation lacks comparative analyses justifying Severus's selection over 
other SV callers, possibly due to its target audience 
\cite{oxford_nanopore_technologies_epi2me_nodate}.

% Esto da posible
% idea de por qué Severus fue integrado en la ultima release de EPI2ME, una 
% plataforma opensoruce tras la cual está Oxford Nanopore con el objetivo de 
% proveer a los wet lab scientist de una interfaz amigable con la que poder 
% analizar sus datos sin necesidades de habilidades bioinformáticas avanzadas, sin 
% empargo (quizás por el publico al que se dirige) no se encontraron en su página 
% web recursos relacionados con comparativas entre SV callers que justifiquen la 
% elección de esta herramienta  

% The open-source EPI2ME platform enhances global access to best-practice 
% analysis workflows, irrespective of bioinformatics expertise. It allows us 
% to deploy, community-standardised software solutions, tailored to the needs of 
% our research projects.

Our evaluation employed minimum argument sets for all callers, relying on 
developer-configured default parameters for complex adjustments. This 
standardized approach may explain the poor performance of Sniffles2 and 
SVision-pro, particularly in detecting large SVs, as default settings might not 
be optimized for such variants. Notably, following our analysis, Sniffles2 
version 2.5 released improvements specifically targeting detection of large 
deletions and duplications ($>$ 50 kb) \cite{noauthor_releases_nodate}. 
Interestingly, the better-performing tools in our analysis (SAVANA and Severus) 
remain available only in preprint servers, while Sniffles2 and SVision-pro are 
published in peer-reviewed journals.

% Las métricas de calling performance posicionan a SAVANA como el mejor de 
% de todos, sin embargo el hecho de que solo sea capaz de correlacionar los 
% breakpints sin llegar a clasificar el tipo de SVs subyacentes es una gran 
% desventaja a la cual se une el muy superior consumo de memoria RAM y tiempo De
% ejecución frente al resto de SV callers. Severus con una precisión ligeramente 
% menor pero mismo recall y menor consumo de RAM, sí que es capaz de identificar 
% el tipo de SV generando además plots para facilitar la comprensión de 
% reordenamientos cromosómicos detectados. 

% La ejecución de los SV callers se llevo a cabo incluyendo minumum set of 
% arguments, lo que conlleva a que muchos parámetros complejos de ajuste de estas 
% herramientas recaigan en valores y opciones preconfiguradas por los 
% desarrolladores de dichas herramientas. Una posible causa del deficiente 
% rendimiento de Sniflles2 y SVision-pro puede ser es la baja idoneidad de estos
% preajustes para la busqueda de SVs muy grandes. Cabe destacar que
% Las versiones de los callers están en constante renovación/actualización y en 
% este sentido con relación a Sniffles2, tras la ejecución de los análisis 
% realizados para este proyecto declaró en el lanzamiento de su versión 2.5 una 
% mejora en la detección de grandes delecciones y duplicaciones >50 kb. No
% obstante, es llamativo el hecho de que tanto Sniffles2 como SVision-pro sean las 
% herramientas ya publicadas en revistas científicas con revisión por
% pares, mientras que SAVANA y Severus solo están disponibles en preprint servers.

\section{Clinical Relevance and Technical Challenges}

VISOR toolkit proved valuable for evaluating SV callers' capability to identify 
characteristic large structural events in Multiple Myeloma using ONT long reads, 
particularly those with diagnostic significance. Through synthetic data 
generation, we successfully validated the detection of two of the most frequent 
chromosomal aberrations in Multiple Myeloma, notably the largest SVs in our 
simulation set, using SAVANA and Severus: the tandem amplification of 1q21+ 
(100 Mb) and an IGH-involving translocation (1,5 Mb). These structural variants, 
currently verified in clinical settings through FISH due to the limitations of 
short-read sequencing assembly, represent critical diagnostic markers that could 
potentially be identified through long-read sequencing approaches.

The inability to simulate certain SVs provides valuable insights into technical 
limitations. While VISOR-HACk module unambiguously inserts all chromosomal 
abnormalities into a FASTA file using BED-formatted instructions and the 
reference genome as a template, challenges emerge in the VISOR-LASeR module,
which generates and aligns reads to the reference genome using minimap2. These 
limitations manifest in two ways: standard read lengths may be insufficient for 
reconstructing certain events, suggesting the potential need for ultra-long read 
protocols capable of generating sequences up to 4 Mb, and minimap2 alignment 
accuracy may be compromised for complex variants. For instance, in the 
copy-paste translocation designed to generate a proximal KRAS duplicate, the 
aligner appears to have defaulted to mapping all reads to the original gene 
position. Similarly, for the cut-paste translocation, only the deletion 
component was detected, while the inverted sequence insertion failed to be 
properly positioned, leaving its corresponding reads unaccounted for in the 
alignment.

% El no haber sido capaz de generar algunas SVs es también bastante interesante.
% El módulo VISOR-HACk inserta inequívocamente todas las anomalías cromosómicas, 
% a través de instrucciones en formato bed, en un fichero fasta que usa el genoma 
% de referencia como base. El "problema" parece estar en el módulo VISOR-LASeR, 
% que se encarga de generar lecturas y alinearlas al genoma de referencia usando 
% minimap2. Por una parte, la longitud de las lecturas puede ser
% demasiado limitada para reconstruir algunos eventos, lo cual en la práctica
% real se traduciría en la necesidad de implementar protocolos para generar 
% ultra long reads, that aims to produce lecturas de up to 4 Mb. Por otra parte y
% de forma no excluyente con lo anterior minimap2 puede no estar alineando 
% correctamente las lecturas; en el caso de la traslocación copy-paste con la 
% que se pretendía generar una segunda copia cercana de KRAS, parece que el 
% el alineador evitado la inserción a optado por alinear todas las lecturas a la 
% altura de la posición original de este gen. De la traslocación cut-paste, solo 
% la delección generada fue detectada, fallando nuevamente en posicionar la 
% inserción aunque en este caso se trata de una inversión y no se aprecia dónde 
% a podido ser reubicadas sus lecturas.

% VISOR toolkit ha mostrado ser útil para interrogar sobre la capacidad de 
% SV Callers de identificar a través de long reads de ONT grandes eventos 
% estructurales característicos y con capacidad diagnóstica de Myeloma Multiple.
% Se generaron datos sintéticos y se lograron identificar en ellos dos de las 
% aberraciones cromosómicas más frecuentes en Myeloma Múltiple que son además
% las más grandes del set de SVs simuladas, la amplificación
% en tandem de 1q21+ (100 Mb ) y una tranlocación que involucra al gen IGH (1,5), 
% en los SV callers SAVANA y Severus. Dichas SVs son actualmente comprobadas en 
% clínica mediante FISH debido la imposibilidad de ensamblarlas mediante 
% secuenciación de lecturas cortas. 

% cromosómicas más caracterísiticas de esta enfermedad
% fueron reconstruidas con datos sintéticos 

% recrear aberraciones cromosómicas 
% Cabe recordar que el módulo VISOR-HACk genera una copia en formato fasta del 
% genoma de referencia que contiene inequívocamente los eventos estructurales que 
% le son ordenados a través de un fichero bed y que el módulo VISOR-LASeR genera 
% lecturas a partir de dicha copia 

\section{Visualization Tools and Challenges}

Initial visualization attempts of synthetic data were made using the 
widely-adopted Integrative Genomics Viewer (IGV) version 2.17.3. However, 
despite using chromosome-specific BAM files generated through the ``bam-splitter'' 
workflow, IGV's performance proved inadequate, exhibiting slow loading times and 
frequent crashes. Through email correspondence, Severus's lead developer shared 
similar experiences with long-read BAM files in IGV, suggesting a workaround of 
generating smaller BAM files containing only the SV regions with 1 Mb upstream 
and downstream sequences. While this approach would have been feasible for 
planned SVs, it proved impractical for investigating additional findings like 
VNTR anomalies due to automation limitations.

GW emerged as a capable alternative, efficiently handling both individual 
chromosome and whole-genome BAM files. Its VCF compatibility created an 
interactive index for rapid SV coordinate navigation, dramatically accelerating 
event verification. However, GW requires terminal-based operation in conjunction 
with an alignment view display, demanding more advanced computational skills 
compared to IGV's graphical user interface \textbf{Figure~\ref{fig:GW_UI}}. 
Nevertheless, GW's developer demonstrated strong responsiveness to error reports 
and feature requests through Github. Our project-specific experience led to 
reporting Conda installation \cite{noauthor_error_nodate} and loading genome
anotation files \cite{noauthor_display_nodate} issues, and requesting vector 
format export capabilities for alignment visualizations 
\cite{noauthor_saving_nodate}.

% En un principio, se trató de realizar la visualización de los datos sintéticos 
% a través del popular y estandarizado Integrative Genomics Viewer 
% (IGV) en su versión 2.17.3, sin embargo el rendimiento de esta herramienta al 
% cargar ficheros BAM individualizados por cromosoma mediante "bam-splitter" 
% workflow fue bastante deficiente debido a que IGV iba demasiado lento y 
% constantly crashes. En conversaciones por mail la desarrolladora principal de 
% Severus compartió esta misma esperiencia trabajando con Long read bam files on
% IGV, y compartió como estrategia para mitigar este problema generate small bams 
% only with the SVs regions and 1mb upstream and downstream. Esta estrategia 
% habría fácilmente abarcable para el set de SVs planificadas pero en el caso de
% allazgos adicionales como los ocurridos en las VNTRs habría requerido 
% demasiado trabajo debido a la imposibilidad de automatización. 

% GW ha demostrado ser una alternativa capaz de permitir visualizar los ficheros 
% BAM de cromosomas sueltos e incluso ficheros de genomas completos, cuya 
% compatibilidad con ficheros VCF con los SV callings para generar un indice 
% interactivo con el que saltar entre coordenadas de las SVs en cuestión de 
% segundos ha agilizado drásticamente la comprobación de los eventos. 
% Sin embargo,la herramienta requiere ser used directly from the terminal en 
% combinación con una alignment view display, lo cual implica tener skills 
% informáticas mucho más avanzadas que las que se necesitan para pulsar los 
% botones de la interfáz grafica de usuario de IGV. No obstante, el desarrollador 
% de GW se mostró bastante abierto ante request de errores y peticiones de 
% posibles mejoras a través de Github; derivado del trabajo específico con la 
% herramienta para este proyecto, se le informó de errores asociados a la 
% instalación de este software a través de Conda, así como se le solicitó como 
% mejora la posibilidad de exportar visualizaciones de los alineamientos en 
% formato vectorial para añadirlos como figuras en este trabajo [link]. 

% GW es la ostia e IGV una mierda ¡Benckmarks!

\section{Future Directions}

The development of robust tools for SV analysis in cancer genomes requires 
extensive testing and validation. While this study evaluated SV detection 
capabilities using synthetic data across different sequencing coverages, 
it primarily demonstrates the value of generating simulated datasets for 
benchmarking existing tools, developing new algorithms, and training machine 
learning models. These computational advances ultimately contribute to 
improving cancer genomics analysis in clinical settings.

% The advancement of cancer patient care requires comprehensive genomic analysis 
% tools capable of identifying complex structural variants. While this study 
% evaluated SV detection capabilities using synthetic data across different 
% sequencing coverages, several critical challenges and opportunities for 
% improvement have emerged that could enhance clinical applications.

A primary challenge lies in identifying structural events that, despite 
existing in the template genome, remained undetected in alignments. The 
unaligned reads in FASTQ format will be valuable for expanding benchmarks to 
include long-read aligners beyond minimap2, enabling more comprehensive tool 
evaluation. This expansion is crucial for improving the reliability of genomic 
analysis in clinical settings.

To enhance the exploration of SVs in cancer, future work should investigate 
several key aspects. First, examining the impact of read length on Sniffles2 and 
SVision-pro's calling capabilities, starting with sizes successfully identified 
in their published work. Second, introducing tumor heterogeneity through 
multiple clone combinations in synthetic sequencing data would better reflect 
real tumor complexities and allow evaluation of how varying VAFs affect SV 
detection capabilities. Third, exploring ultra-long read protocol parameters 
could help reconstruct previously undetected SVs. Based on our findings and 
resource efficiency considerations, these tests could focus on 30x and 50x 
sequencing coverages, aligning with practical clinical sequencing depths.

Data collection automation represents another critical area for improvement in 
clinical implementation. Currently, computational performance statistics are 
manually extracted from Slurm logs, a time-consuming process that could be 
streamlined through automated scripts. In anticipation of future evaluations, 
we have requested VISOR toolkit enhancements to log all parameters, both 
specified and default, facilitating automated tracking of simulation 
characteristics \cite{noauthor_add_nodate}. Additionally, The implementation of 
Truvari, a comprehensive toolkit for benchmarking, merging, 
and annotating structural variants from VCF files, would enhance the workflow
\cite{english_truvari_2022}. Its capabilities make it 
particularly suitable for analyzing tumor genome evolution through longitudinal 
sequencing and SV calling analysis.

Continuous communication with tool developers remains essential for optimizing 
clinical applications, sharing synthetic data experiences, and facilitating 
potential improvements. In this context, frequent requests for GW enhancements, 
particularly regarding data visualization and high-quality figure export 
capabilities, will support better clinical result interpretation and 
documentation.

Most critically, the experience gained through synthetic data analysis must be 
validated on ONT-sequenced tumor-normal paired patient samples, integrating 
structural variation analysis with SNVs and methylation data. This comprehensive 
genomic characterization approach aims to provide clinicians with more accurate 
and complete information for personalized cancer treatment decisions, ultimately 
improving patient outcomes through better-informed therapeutic strategies.

% This study evaluated the detection capabilities of various structural variants 
% across different sequencing coverages using synthetic data. A key challenge 
% ahead lies in identifying structural events that, despite existing in the 
% template genome, remained undetected in alignments. The unaligned reads in 
% FASTQ format will be valuable for expanding benchmarks to include long-read 
% aligners beyond minimap2, enabling comprehensive tool evaluation.

% To investigate Sniffles2 and SVision-pro's underwhelming performance, future 
% work should examine the impact of read length on their calling capabilities. 
% This could be achieved by generating new simulation sets maintaining the same 
% SVs but varying their lengths, starting with sizes successfully identified in 
% their published work. Additionally, introducing tumor heterogeneity through 
% multiple clone combinations in synthetic sequencing data would allow evaluation 
% of how varying Variant Allele Frequencies affect SV detection capabilities. 
% Furthermore, simulating parameters equivalent to ultra-long read protocols 
% could help assess their potential for reconstructing SVs that remained 
% undetected in current alignments. Based on our current findings and considering 
% computational and storage efficiency, these tests could focus on 30x and 50x 
% sequencing coverages.

% To investigate Sniffles2 and SVision-pro's underwhelming performance, future 
% work should examine the impact of read length on their calling capabilities. 
% This could be achieved by generating new simulation sets maintaining the same 
% SVs but varying their lengths; empezando desde tamaños asquibles que han 
% mostrado ser capaces de identificar según sus publicaciones. Additionally, 
% introducing tumor heterogeneity through multiple clone combinations in synthetic 
% sequencing data would allow evaluation of how varying Variant Allele Frequencies 
% affect SV detection capabilities. Otro elemento interesante sería secuenciar con
% parámetros equivalentes a secuenciar con protocolos de ultra long reads y ver
% si esto sirve para reconstruir las SVs que no han podido obtenerse en los 
% alineamientos.Based on our current findings and considering 
% computational and storage efficiency, these tests could focus on 30x and 50x 
% sequencing coverages.

% % To investigate Sniffles2 and SVision-pro's underwhelming performance, future 
% % work should examine the impact of read length on their calling capabilities. 
% % This could be achieved by generating new simulation sets maintaining the same 
% % SVs but varying their lengths. Otra variable interesante a introducir es la 
% % heterogeneidad tumoral combinando variedad de clones a los datos sintéticos de
% % secuenciación para evaluar la variación de las Variant Allele Frecuencies en 
% % la capacidad de detección en los  SV callers. Based on our current findings and 
% % considering computational and storage efficiency, these tests could focus on 30x 
% % and 50x sequencing coverages.

% Data collection automation represents another critical area for improvement. 
% Currently, computational performance statistics from all tools are manually 
% extracted from SLURM logs, a time-consuming process that could be streamlined 
% through automated scripts. Como anticipación a estas posibles pruebas, se 
% solicitó la mejora de VISOR toolkit consistente en imprimir en los ficheros log 
% todos los parámetros, sean especificados o por defecto, para automatizar y 
% mantener registro de las características de cada simulación. Based on our current 
% findings and considering computational and storage efficiency, these tests could 
% focus on 30x and 50x sequencing coverages.Additionally, implementing Truvari, a 
% toolkit for benchmarking, merging, and annotating Structural Variants from VCF 
% files, could enhance analysis capabilities and potentially support tumor genome 
% evolution studies through sequencing and SV calling.

% Ongoing communication with tool developers remains essential for both receiving 
% guidance on optimal tool usage and sharing synthetic data experiences to 
% facilitate potential improvements and bug fixes. In this context, continued 
% requests for GW enhancements, particularly regarding data visualization and 
% high-quality figure export capabilities, will be valuable for future analyses.

% % acidad de calling de ambas herramientas. Para ello, 
% % una aproximación interesante sería generar nuevos sets de simulaciones con las
% % mismas SVs, donde la diferencia estaría en la longitud de estos eventos. Para
% % estas pruebas, en base a los resultados previos y el ahorro significativo en
% % computación y almacenamiento los datos sintéticos podrían ser generados 
% % únicamente a coberturas de secuenciación 30x y 50x.

% % La automatización en la recolección de los datos es otro elemento crítico en 
% % el que se necesita trabajar para poder seguir avanzando en la generación de los
% % benchmarks. Las estadísticas de rendimiento computacional de todas las 
% % herramientas utilizadas en este trabajo fueron recogidas manualmente de los 
% % logs generados por SLURM, lo cual supuso un tiempo bastante considerable que 
% % podría ahorrarse a través de la generación de scripts que se encarguen de esta 
% % tarea. Otro elemento que sería interesante de implementar sería Truvari, un 
% % toolkit for benchmarking, merging, and annotating Structural Variants a partir 
% % de ficheros VCF, el cual en base a sus funcionalidades podría servir también
% % para analizar la evolución de genomas tumorales a partir de secuenciación y
% % SV calling.

% % El contacto con los desarrolladores de todas las herramientas será crucial
% % tanto para recibir feedback de cómo hacer usos adecuados de las herramientas 
% % como compartir la experiencia con los datos sintéticos de cara a posibles 
% % mejoras y correcciones de errores. En relación con esto será interesante seguir
% % solicitando mejoras de GW, especialmente aquellas relacionadas con visualización
% % de los datos que luego puedan ser exportados to get high quality figures.En este trabajo a partir de los datos sintéticos se evaluó la capacidad de 
% % detección de diferentes tipos de grandes eventos estructurales ante diferentes
% % coverturas de secuenciación. Un próximo reto interesante será conseguir
% % que aquellos eventos estructurales que aun existiendo el genoma molde de las
% % lecturas no consiguieron avistarse luego en los alineamientos. Para esto serán
% % de gran ayuda las unaligned reads in FASTQ format, que permitirán añadir a los
% % benchmarks otros alineadores de lecturas largas distintos de minimap2 para 
% % evaluar también este tipo de herramientas. 

% % Para profundizar en la búsqueda de razones de los malos resultados de Sniffles2
% % y SVision-pro, será interesante estudiar el efecto de la longitud de las 
% % lecturas sobre la cap

% Most importantly, the experience gained through synthetic data analysis needs to 
% be applied and validated on ONT-sequenced tumor-normal paired patient samples, 
% integrating structural variation analysis with SNP and methylation data.

% % Por último, pero lo más importante, se requiere utilizar y validar la 
% % experiencia adquirida con datos sintéticos en el análisis de tumour-normal
% % paired samples of patients  secuenciados mediante ONT, integrando la variación 
% % estructural junto con los datos de SNPs y metilación.



