\chapter{Objetives}

The primary aim of this research is to evaluate cutting-edge long-read 
sequencing approaches to track and analyze somatic structural variations across 
cancer genomes evolution. To this end, we have established the following 
specific objectives:

\begin{enumerate}

    \item Assess the technical feasibility of detecting large-scale structural 
    variants using synthetic long-read sequencing data that simulates current 
    ONT flow cell protocols.
    
    \item Evaluate and compare current SV callers, considering both their 
    detection accuracy and computational efficiency, to identify 
    optimal solutions for analysis workflows.
    
    \item Validate the possibility of detecting clinically relevant Multiple 
    Myeloma markers using long-read sequencing approaches, identifying both 
    their potential and current limitations for clinical applications.
    
\end{enumerate}

% \item Evaluate and benchmark the accuracy of SV detection algorithms through 
% the generation of synthetic long-read sequencing datasets containing diverse 
% SV profiles across multiple length scales.

% \item Assess the performance of top-performing structural variant callers 
% using real tumor-normal paired sequencing datasets, sequenced specifically 
% for this work or obtained from public repositories.

% \item Develop a comprehensive strategy for reliable SV analysis in cancer 
% genomics that combines the most accurate callers and potential auxiliary 
% tools, addressing the current absence of gold standard methodologies.