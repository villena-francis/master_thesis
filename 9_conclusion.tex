\chapter{Conclusions}

Experimental data is essential for creating and improving computational methods, 
whose outputs not only provide new knowledge but can also serve as a foundation 
for refining future experiments. Synthetic data can catalyze this cycle, 
generating information faster and with fewer resource expenditures. Based on the 
results of this work, we can conclude the following:

\begin{enumerate}

    \item Long-read WGS data obtainable with a single ONT PromethION flow cell 
    enables the reconstruction of large-scale SVs, including insertions, deletions, 
    translocations, and inversions spanning complete chromosomal bands.
    
    \item Severus emerges as a balanced solution for SV detection, combining 
    comprehensive variant classification capabilities with efficient resource 
    utilization, making it suitable for routine analysis workflows.
    
    \item Long-read sequencing has the potential detect of clinically 
    significant MM markers, promising to enhance current diagnostic capabilities 
    despite some technical limitations in SV detection that remain to be addressed.

\end{enumerate}

These findings provide valuable guidance for SV detection implementation while 
establishing a framework for synthetic data generation that can be used to 
benchmark tools and train future machine learning models. This work also 
highlights areas requiring further development in long-read sequencing analysis.